%!TEX root = ./report.tex
As previously mentioned Imp uses a hoare triple to reason about the behaviour of the above commands. Due to the change to assertions to now also make claims about heaps a change to the definition of this hoare triple is needed. Initially the hoare triples way to express claims about commands was as follows:
\\
\\
\textit{If a command c is started in a state satisfying an assertion P then the resulting state will satisfy an assertion Q, if c terminates.}
\\
\\
While this essentially still holds there are certain parameters added that the above no longer reasons about. As discussed in the secion \ref{sec:error_semantics}, some of the added commands can fail, which will result in an faulty state. This is an issue, since the hoare triple claims that resulting state is guaranteed to satisfy the post-condition, which it cannot do if it is faulty. As such we need our hoare triple to also express that the command c does not result in an error state.
\\
\\
\textit{If a command c is started in a state st satisfying an assertion P, and c does not fail in st, then the resulting state will satisfy an assertion Q, if c terminates.}
\\
\\
\paragraph{Safety Monotonicity}