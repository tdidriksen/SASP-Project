%!TEX root = ./report.tex
\subsection{Definition of the Hoare Triple}
\label{sec:hoare_triple}
As previously mentioned, specifications of Imp programs are given using Hoare triples. Due to the new assertion type presented in Section \ref{sec:assertions_heaps}, a change of the definition of this hoare triple is needed accordingly. The basic Hoare triple is defined as follows:
\\
\\
\textit{If a command c is started in a state satisfying an assertion P, then the resulting state will satisfy an assertion Q, if c terminates.}\todo{Add reference to definition.}
\\
\\
While this essentially still holds for the extended Imp specifications, this definition is insufficient. As discussed in Section \ref{sec:error_semantics}, some of the heap-manipulating operations can evaluate to a faulty state. This is an issue, since the Hoare triple claims that the resulting state is guaranteed to satisfy the post-condition. With the above definition, this can no longer be guaranteed if the resulting state is faulty. As such, we need our Hoare triple to  express that the command c does not result in an error state.
\\
\\
\textit{If a command c is started in a state st satisfying an assertion P, and c does not fail in st, then the resulting state will satisfy an assertion Q, if c terminates.}
\\
\\
To express the idea of a command not failing in a state, we introduce the notion of safety. A command is safe in a state {\it st} if the command, starting in {\it st}, does not evaluate to None. We therefore define safety as follows:

\[safe\;c\;st\;: \;\neg \;(<c,\;st>\;\Downarrow None)\]

Having formalized the a safety predicate, we can now formulate a new definition of our Hoare triple. This gives us the following definition of a Hoare triple:

\[
\forall st.\; P\;st \impl safe\;c\;st \land \forall st'. <c,\;st>\;\Downarrow Some\; st' \impl Q\;st'
\]	