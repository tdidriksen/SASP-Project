%!TEX root = ./report.tex
As previously mentioned Imp uses a hoare triple to reason about the behaviour of the commands. Due to the change to assertions to now also make claims about heaps, a change to the definition of this hoare triple is needed. Initially the hoare triples way to express claims about commands was as follows:
\\
\\
\textit{If a command c is started in a state satisfying an assertion P, then the resulting state will satisfy an assertion Q, if c terminates.}
\\
\\
While this essentially still holds there are certain parameters added that the above no longer reasons about. As discussed in the secion \ref{sec:error_semantics}, some of the added commands can fail, which will result in an faulty state. This is an issue, since the hoare triple claims that resulting state is guaranteed to satisfy the post-condition, which it cannot do if it is faulty. As such we need our hoare triple to also express that the command c does not result in an error state.
\\
\\
\textit{If a command c is started in a state st satisfying an assertion P, and c does not fail in st, then the resulting state will satisfy an assertion Q, if c terminates.}
\\
\\

To express the idea of a command not failing in a state we introduce the notion of safety. A command is safe in a state if the command does not evaluate to None. We therefore  define safety as follows:

\[safe\;c\;st\;: \;\neg \;(c \;\textbackslash{}\;st\;\Downarrow None)\]

With this we can formulate a new definition of our hoare triples. As stated above the resulting state of c satisfies Q if c is safe in st. Since the case where c is not safe in st never can evaluate to true, this is expressed with conjunction rather than implication\todo{Maybe omit}. This gives us the following definition for hoare triples:

\[
\forall st.\; P\;st \impl safe\;c\;st \land \forall st'. \;c\;\textbackslash{}\;st\;\Downarrow Some\; st' \impl Q\;st'
\]	