%!TEX root = ./report.tex
\section{Solution}
\todo{mentioned only nats here, or maybe even earlier}
\subsection{Programming with Mutable State}
\label{sec:programming_with_mutable_state}
For our example mentioned in the introduction to be implementable in Imp, we need to introduce the idea of a mutable state. Originally the state in Imp is managed by a stack, which is a function from variables to values. We extend this state by adding a heap, which will provide us with a mapping between addresses and value. These values represent the mutable structures. As mentioned earlier our heap only handles natural numbers, and for the sake of simplicity, we will consider heaps to be infinite. We define heaps as a partially applied function from addresses ($nat$) to values ($nat$):

\[
\heap \eqdef a\rightharpoonup v, \; where\; a, \;v \in \mathbb{N}
\] 

A heap alone, however, does not allow our example to be implemented. In section \ref{sec:heap_operations} we will discuss the implementation of operations on this heap and their integration into Imp. 

%!TEX root = ./report.tex
\subsection{Operations on the Heap}
\label{sec:heap_operations}
We now define syntax and operational semantics for the operations added to the Imp language for manipulating the heap. Four operations are needed: Read (or lookup), Write (or mutate), Allocate, and Deallocate (or dispose).
\paragraph{Syntax.}
The Imp syntax for each of the operations is shown in Figure \ref{fig:heap_semantics} as the first part of the \textless \textgreater -notation.
\paragraph{Functions manipulating the heap.}
\todo{Consider moving this section into 'Programming with mutable state'}
The following heap-manipulating functions underlying the implementation of the heap operations in Imp are used in the semantics presented in Figure \ref{fig:heap_semantics}. The definitions of {\it a} and {\it v} are as presented in Section \todo{Add reference to 'Programming with mutable state' here}.
\begin{itemize}
\item {\it write} $\eqdef$ {\it a} $\rightharpoonup$ {\it v} $\rightharpoonup$ {\it heap}
\item {\it dealloc} $\eqdef$ {\it a} $\rightharpoonup$ {\it heap}
\item {\it alloc} $\eqdef$ {\it a} $\rightharpoonup$ {\it n} $\rightharpoonup$ {\it heap}
\end{itemize}
The second parameter of the {\it alloc} function, {\it n}, is a natural number specifying the number of cells to allocate.

\subsubsection{Semantics}
\begin{figure}
\[
    \infrule[Read]{
       heap(a) = n
    }{
       < (X\;<\sim\; [ a ]),\;Some(s,\;h) > \; \Downarrow Some\;([s\;|\;X:n],\;h)
    }
\]
\\
\[
    \infrule[Write]{
       \exists e. heap(a) = e
    }{
       < ([ a ]\;<\sim\; b),\;Some(s,\;h) > \; \Downarrow Some\;([s\;|\;X:a],\;write\;a\;b\;h)
    }
\]
\\
\[
    \infrule[Deallocate]{
       \exists e. heap(a) = e
    }{
       < (DEALLOC\;a),\;Some(s,\;h) > \; \Downarrow Some\;(s,\;dealloc\;a\;h)
    }
\]
\\
\[
    \infrule[Allocate]{
       \neg \exists e. heap(a) = e \;\;\;\;\;\;\;\;\; \forall n.\; (n > a \wedge n \leq a+n) \implies \neg \exists e'. heap(n) = e'
    }{
       < (X=ALLOC\;n),\;Some(s,\;h) > \; \Downarrow Some\;([s\;|\;X:a],\;alloc\;a\;n\;h)
    }
\]
\caption{Semantics for the heap operations}
\label{fig:heap_semantics}
\end{figure}
The semantics for the heap operations are given in Figure \ref{fig:heap_semantics}. There are a few interesting points to note here. Firstly, the semantics of allocation and deallocation are asymmetric. This asymmetry becomes evident when the input to the two underlying functions are examined: {\it alloc} accepts a number {\it n} of cells to allocate, while {\it dealloc} accepts a single address to deallocate. Therefore, it is possible to encapsulate the allocation of an arbitrary number of memory cells in a single allocation operation, while deallocating {\it n} cells from the heap has to be done in {\it n} individual deallocation operations. This limitation makes deallocation easier to reason about, but places a heavier burden of memory management on the user of the language.

Secondly, locating free addresses in the address space for the allocation operation is not a part of the allocation semantics, although the semantics do state that these addresses must be consecutive not already allocated. \todo{Perhaps this point belongs in a different section} Here, we simply assume that such consecutive and free addresses exist in the heap, and leave it up to the soundness proof of the corresponding Hoare rule (see Section \todo{Add reference to Hoare rules section here}) to prove this fact.

Lastly, it should be noted that we have no notion of void types and do not provide any measures for ignoring evaluated values. Consequently, providing a variable in which to store the result of a read or allocation operation is required. For allocation, this also has the benefit of safeguarding against allocating unreachable memory, which would be unfortunate in an environment without garbage collection.

\todo{The address space of the heap is infinite! Write this somewhere}
\todo{We allocate cells with 0! Write this somewhere}

\subsubsection{Error Semantics}
\label{sec:error_semantics}
An interesting consequence of adding an addressable state to a programming language is the possibility of writing programs that evaluate to a faulty state. Contrary to the basic version of the Imp language, an addressable state enables the user to write programs that type check but evaluate to an erroneous state: When reading, writing, or deallocating, the program might end up in a faulty state if attempting to access an inactive address on the heap. Note that because we assume that the address space of the heap is infinite, allocation cannot fail. We present the error semantics in Figure \ref{fig:heap_error_semantics}.
\begin{figure}
\[
    \infrule[ReadError]{
       \neg \exists e. heap(a) = e
    }{
       < (X\;<\sim\; [ a ]),\;Some(s,\;h) > \; \Downarrow None
    }
\]
\\
\[
    \infrule[WriteError]{
       \neg \exists e. heap(a) = e
    }{
       < ([ a ]\;<\sim\; b),\;Some(s,\;h) > \; \Downarrow None
    }
\]
\\
\[
    \infrule[DeallocateError]{
       \neg \exists e. heap(a) = e
    }{
       < (DEALLOC\;a),\;Some(s,\;h) > \; \Downarrow None
    }
\]
\caption{Error semantics for the Read, Write, and Deallocate operations}
\label{fig:heap_error_semantics}
\end{figure}
%!TEX root = ./report.tex
\subsection{Creating Assertions}
\label{sec:assertions_heaps}
We can create specifications from the assertion logic presented in Section \ref{sec:separation_logic} for Imp programs involving heaps. Following our definition of the logic, we must prove that any assertion used in a specification is closed under the equivalence of heaps. Consequently, and contrary to the basic version of Imp, we cannot simply state our assertions as functions on a state, such as shown in Figure \ref{fig:assertions_basic_imp}.

\begin{figure}
\[
	\infrule{}
		{
		\triple
			{\;fun\;st => st\;X = aeval\;st\;(ANum\;0)\;}
			{\;SKIP\;}
			{\;fun\;st => st\;X = aeval\;st\;(ANum\;0)\;}
		}
\]
\caption{An example of a specification in the basic version of Imp.}
\label{fig:assertions_basic_imp}
\end{figure}

However, because of the implementation of the logic, we do not want to model our assertions as explicit functions inside the specifications. Firstly, the explicit function definition constitutes unnecessary clutter in our specifications. Secondly, exactly because our connectives are pointwise liftings, the state and the heap need not be mentioned in the specifications at all. Instead, we define and prove assertions in terms of the heap and the state. An example analogous to the one in Figure \ref{fig:assertions_basic_imp} is shown in Figure \ref{fig:assertions_extended_imp}.

\begin{figure}
\[
	\infrule{}
		{
		\triple
			{\;aexp\_eq\;(AId X)\;(ANum\;0)\;}
			{\;SKIP\;}
			{\;aexp\_eq\;(AId X)\;(ANum\;0)\;}
		}
\]
\caption{An example of a specification in the extended version of Imp.}
\label{fig:assertions_extended_imp}
\end{figure}

The assertion {\it aexp\_eq} in Figure \ref{fig:assertions_extended_imp} is defined as a function \heap $\rightharpoonup$ {\it state} $\rightharpoonup$ {\it Prop} closed under the equivalence of heaps. All expressions shown inside preconditions and postconditions in the following sections are modeled in this manner, including the points-to predicate.

Naturally, as all the hoare rules from the basic version of Imp presented in Section \ref{sec:background_imp} were not implemented in our assertion logic, these have been reimplemented in this logic.

%operators are pointwise liftings
\todo{Add something about the implications of changing the assertion logic for the existing Imp.}
%!TEX root = ./report.tex
As previously mentioned Imp uses a hoare triple to reason about the behaviour of the commands. Due to the change to assertions to now also make claims about heaps, a change to the definition of this hoare triple is needed. Initially the hoare triples way to express claims about commands was as follows:
\\
\\
\textit{If a command c is started in a state satisfying an assertion P, then the resulting state will satisfy an assertion Q, if c terminates.}
\\
\\
While this essentially still holds there are certain parameters added that the above no longer reasons about. As discussed in the secion \ref{sec:error_semantics}, some of the added commands can fail, which will result in an faulty state. This is an issue, since the hoare triple claims that resulting state is guaranteed to satisfy the post-condition, which it cannot do if it is faulty. As such we need our hoare triple to also express that the command c does not result in an error state.
\\
\\
\textit{If a command c is started in a state st satisfying an assertion P, and c does not fail in st, then the resulting state will satisfy an assertion Q, if c terminates.}
\\
\\

To express the idea of a command not failing in a state we introduce the notion of safety. A command is safe in a state if the command does not evaluate to None. We therefore  define safety as follows:

\[safe\;c\;st\;: \;\neg \;(c \;\textbackslash{}\;st\;\Downarrow None)\]

With this we can formulate a new definition of our hoare triples. As stated above the resulting state of c satisfies Q if c is safe in st. Since the case where c is not safe in st never can evaluate to true, this is expressed with conjunction rather than implication\todo{Maybe omit}. This gives us the following definition for hoare triples:

\[
\forall st.\; P\;st \impl safe\;c\;st \land \forall st'. \;c\;\textbackslash{}\;st\;\Downarrow Some\; st' \impl Q\;st'
\]	
%!TEX root = ./report.tex
\subsection{Hoare Rules}
\label{sec:hoare_rules}
% All the rules have been proven sound in Coq.
This section describes Hoare rules for each of the four operations presented in Section \ref{sec:heap_operations}. All of the rules have been proven sound in Coq. Note that these rules are local\,\cite{Reynolds02}, and thus rely on the frame rule (see Section \ref{sec:frame_rule}) for reasoning about non-local specifications.
\subsubsection{Read}
\[
	\infrule[Read]{}
		{
		\triple
			{e\mapsto e'}
			{\;X<\sim\lbrack\;e\;\rbrack\;}
			{\exists v.\;(e\;\mapsto\;e')\subst{v}{X}\;\land\;(X=e')\subst{v}{X}}
		}
\]


\subsubsection{Write}
\[
	\infrule[Write]{}
		{
		\triple
			{e\mapsto-}
			{\;\lbrack\;e\;\rbrack\;<\sim\;e'\;}
			{e\mapsto e'}
		}
\]


\subsubsection{Allocate}
\[
	\infrule[Allocate]{}
		{
		\triple
			{\;emp\;}
			{\;X\;\&=ALLOC\;n\;}
			{\exists a. X=a\;\land\;\odot _{i=a}^{a+n-1} i\mapsto0}
		}
\]
The allocation rule allocates {\it n} memory cells as specified by the parameter to {\it ALLOC}, and lets the variable {\it X} point to the first of these cells. As it is the case in other programming languages, such as Java\,\cite{JavaDataTypes}, we allocate memory cells with a default value of zero. We do not have a separate notion of a {\it null} pointer, so pointing to zero is equivalent to pointing to {\it null}. Interpreting a value of zero as either a concrete `0' or a null pointer is up to the program.
% We allocate with 0
% We can always find an address

\subsubsection{Deallocate}
\[
	\infrule[Deallocate]{}
		{
		\triple
			{e\mapsto-}
			{\;DEALLOC\;e\;}
			{emp}
		}
\]

\subsection{Auxiliary Theorems}
%!TEX root = ./report.tex
\paragraph{Safety Monotonicity}
%!TEX root = ./report.tex
\paragraph{Rules of Consequence}
The precondition or postcondition of a Hoare rule might at times differ from the ones needed when proving a program. While the condition in our goal might be logically equivalent to one in a given rule, their differences prevents them from being unified. To accommodate this difference, Hoare logic introduces the rules of consequence\,\cite{Hoare69anaxiomatic}. They state that any precondition $P$ can be substituted for a weaker precondition $P'$, provided that $P$ entails $P'$, while any postcondition $Q$ can be substituted for a stronger postcondition $Q'$, provided that $Q'$ entails $Q$. These rules are shown in Figure \ref{fig:rules_of_consequence}.

\begin{figure}
\[
	\infrule{P\entails P' \;\; \triple{P'}{\;c\;}{Q}} {\triple{P}{\;c\;}{Q}} 
	\;\;\;\;\;\; 
	\infrule{Q'\entails Q \;\; \triple{P}{\;c\;}{Q'}} {\triple{P}{\;c\;}{Q}}
\]
\caption{The rules of consequence.}
\label{fig:rules_of_consequence}
\end{figure}

While these rules have not been changed by our changes to states in Imp, the change to assertions have changed how these rules are proven.
