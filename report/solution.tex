%!TEX root = ./report.tex
\section{Solution}
\todo{mentioned only nats here, or maybe even earlier}
\subsection{Programming with Mutable State}
\label{sec:programming_with_mutable_state}
For our example mentioned in the introduction to be implementable in Imp, we need to introduce the idea of a mutable state. Originally the state in Imp is managed by a stack, which is a function from variables to values. We extend this state by adding a heap, which will provide us with a mapping between addresses and value. These values represent the mutable structures. As mentioned earlier our heap only handles natural numbers, and for the sake of simplicity, we will consider heaps to be infinite. We define heaps as a partially applied function from addresses ($nat$) to values ($nat$):

\[
\heap \eqdef a\rightharpoonup v, \; where\; a, \;v \in \mathbb{N}
\] 

A heap alone, however, does not allow our example to be implemented. In section \ref{sec:heap_operations} we will discuss the implementation of operations on this heap and their integration into Imp. 

%!TEX root = ./report.tex
\subsection{Operations on the Heap}
We now define syntax and operational semantics for the operations added to the Imp language for manipulating the heap. Four operations are needed: Read (or lookup), Write (or mutate), Allocate, and Deallocate (or dispose).
\subsubsection{Syntax} \todo{Show the syntax for each of the commands here. Coqdoc might be of use.}
\subsubsection{Semantics}
\todo{A figure here with the semantics for read, write, allocate, and deallocate in sexy logic notation.}
There are a few interesting points to note here. Firstly, the semantics of allocation and deallocation are asymmetric. This asymmetry becomes evident when the input to the two operations are examined: allocation accepts a (natural) number of cells to allocate, while deallocation accepts a single address. Therefore, it is possible to encapsulate the allocation of an arbitrary number of memory cells in a single allocation operation, while deallocating {\it n} cells from the heap has to be done in {\it n} individual deallocation operations. This limitation makes deallocation easier to reason about, but places a heavier burden of memory management on the user of the language.

Secondly, locating free addresses in the address space for the allocation operation is not a part of the allocation semantics, although the semantics do state that these addresses must be consecutive not already allocated. \todo{Perhaps this point belongs in a different section} Here, we simply assume that such consecutive and free addresses exist in the heap, and leave it up to the soundness proof of the corresponding Hoare rule (see Section \todo{Add reference to Hoare rules section here}) to prove this fact.

Lastly, it should be noted that we have no notion of void types and do not provide any measures for ignoring evaluated values. Consequently, providing a variable in which to store the result of a read or allocation operation is required. For allocation, this also has the benefit of safeguarding against allocating unreachable memory, which would be unfortunate in an environment without garbage collection.

\todo{The address space of the heap is infinite! Write this somewhere}
\todo{We allocate cells with 0! Write this somewhere}

\subsubsection{Error Semantics}
An interesting consequence of adding a an addressable state to a programming language is the possibility of writing programs that evaluate to a faulty state. Contrary to the basic version of the Imp language, an addressable state enables the user to write programs that type check but evaluate to an erroneous state: When reading, writing, or deallocating, the program might end up in a faulty state if attempting to access an inactive address on the heap. Note that because we assume that the address space of the heap is infinite, allocation cannot fail. \todo{Add some more juice to this section.}
%!TEX root = ./report.tex
\subsection{Creating Assertions}
\label{sec:assertions_heaps}
We can create specifications from the assertion logic presented in Section \ref{sec:separation_logic} for Imp programs involving heaps. Following our definition of the logic, we must prove that any assertion used in a specification is closed under the equivalence of heaps. Consequently, and contrary to the basic version of Imp, we cannot simply state our assertions as functions on a state, such as shown in Figure \ref{fig:assertions_basic_imp}.

\begin{figure}
\[
	\infrule{}
		{
		\triple
			{\;fun\;st => st\;X = aeval\;st\;(ANum\;0)\;}
			{\;SKIP\;}
			{\;fun\;st => st\;X = aeval\;st\;(ANum\;0)\;}
		}
\]
\caption{An example of a specification in the basic version of Imp.}
\label{fig:assertions_basic_imp}
\end{figure}

However, because of the implementation of the logic, we do not want to model our assertions as explicit functions inside the specifications. Firstly, the explicit function definition constitutes unnecessary clutter in our specifications. Secondly, exactly because our connectives are pointwise liftings, the state and the heap need not be mentioned in the specifications at all. Instead, we define and prove assertions in terms of the heap and the state. An example analogous to the one in Figure \ref{fig:assertions_basic_imp} is shown in Figure \ref{fig:assertions_extended_imp}.

\begin{figure}
\[
	\infrule{}
		{
		\triple
			{\;aexp\_eq\;(AId X)\;(ANum\;0)\;}
			{\;SKIP\;}
			{\;aexp\_eq\;(AId X)\;(ANum\;0)\;}
		}
\]
\caption{An example of a specification in the extended version of Imp.}
\label{fig:assertions_extended_imp}
\end{figure}

The assertion {\it aexp\_eq} in Figure \ref{fig:assertions_extended_imp} is defined as a function \heap $\rightharpoonup$ {\it state} $\rightharpoonup$ {\it Prop} closed under the equivalence of heaps. All expressions shown inside preconditions and postconditions in the following sections are modeled in this manner, including the points-to predicate.

Naturally, as all the hoare rules from the basic version of Imp presented in Section \ref{sec:background_imp} were not implemented in our assertion logic, these have been reimplemented in this logic.

%operators are pointwise liftings
\todo{Add something about the implications of changing the assertion logic for the existing Imp.}
%!TEX root = ./report.tex
\subsection{Definition of the Hoare Triple}
\label{sec:hoare_triple}
As previously mentioned, specifications of Imp programs are given using Hoare triples. Due to the new assertion type presented in Section \ref{sec:assertions_heaps}, a change of the definition of this hoare triple is needed accordingly. The basic Hoare triple is defined as follows:
\\
\\
\textit{If a command c is started in a state satisfying an assertion P, then the resulting state will satisfy an assertion Q, if c terminates.}\todo{Add reference to definition.}
\\
\\
While this essentially still holds for the extended Imp specifications, this definition is insufficient. As discussed in Section \ref{sec:error_semantics}, some of the heap-manipulating operations can evaluate to a faulty state. This is an issue, since the Hoare triple claims that the resulting state is guaranteed to satisfy the post-condition. With the above definition, this can no longer be guaranteed if the resulting state is faulty. As such, we need our Hoare triple to  express that the command c does not result in an error state.
\\
\\
\textit{If a command c is started in a state st satisfying an assertion P, and c does not fail in st, then the resulting state will satisfy an assertion Q, if c terminates.}
\\
\\
To express the idea of a command not failing in a state, we introduce the notion of safety. A command is safe in a state {\it st} if the command, starting in {\it st}, does not evaluate to None. We therefore define safety as follows:

\[safe\;c\;st\;: \;\neg \;(<c,\;st>\;\Downarrow None)\]

Having formalized the a safety predicate, we can now formulate a new definition of our Hoare triple. This gives us the following definition of a Hoare triple:

\[
\forall st.\; P\;st \impl safe\;c\;st \land \forall st'. <c,\;st>\;\Downarrow Some\; st' \impl Q\;st'
\]	
%!TEX root = ./report.tex
\subsection{Hoare Rules for the Heap Operations}
\label{sec:hoare_rules}
% All the rules have been proven sound in Coq.
This section describes Hoare rules for each of the four operations presented in Section \ref{sec:heap_operations}. All of the rules have been proven sound in Coq. Note that these rules are local\,\cite{Reynolds02}, and thus rely on the frame rule (see Section \ref{sec:frame_rule}) for reasoning about non-local specifications.
\subsubsection{Read}
\[
	\infrule[Read]{}
		{
		\triple
			{e\mapsto e'}
			{\;X<\sim\lbrack\;e\;\rbrack\;}
			{\exists v.\;(e\;\mapsto\;e')\subst{v}{X}\;\land\;(X=e')\subst{v}{X}}
		}
\]
The Read rule reads the value at address {\it e}, {\it e'}, onto the stack. Consequently, all previous occurrences of {\it X} must be substituted with the old value of {\it X}, here denoted by the existential variable {\it v}, to avoid corrupting previous definitions.

\subsubsection{Write}
\[
	\infrule[Write]{}
		{
		\triple
			{e\mapsto-}
			{\;\lbrack\;e\;\rbrack\;<\sim\;e'\;}
			{e\mapsto e'}
		}
\]
The Write rule destructively updates an address on the heap. Importantly, the precondition requires the updated address to be active, i.e. it must exist on the heap beforehand. If one wishes to write to a non-active address, the address must be allocated first.
\todo{Write somewhere that all addresses are evaluated on the stack.}

\subsubsection{Allocate}
\[
	\infrule[Allocate]{}
		{
		\triple
			{\;emp\;}
			{\;X\;\&=ALLOC\;n\;}
			{\exists a. X=a\;\land\;\odot _{i=a}^{a+n-1} i\mapsto0}
		}
\]
The Allocate rule allocates {\it n} memory cells as specified by the parameter to {\it ALLOC}, and lets the variable {\it X} point to the first of these cells. As it is the case in other programming languages, such as Java\,\cite{JavaDataTypes}, we allocate memory cells with a default value of zero. We do not have a separate notion of a {\it null} pointer, so pointing to zero is equivalent to pointing to {\it null}. Interpreting a value of zero as either a concrete `0' or a {\it null} pointer is up to the program.
% We allocate with 0
% We can always find an address

\subsubsection{Deallocate}
\[
	\infrule[Deallocate]{}
		{
		\triple
			{e\mapsto-}
			{\;DEALLOC\;e\;}
			{emp}
		}
\]
The Deallocate rule removes an active address from the heap. As reflected in the semantics presented in Section \ref{sec:heap_operations}, deallocation is asymmetric with respect to allocation.

\subsection{Frame rule and Rules of Consequence}\todo{Consider revising title}
%!TEX root = ./report.tex
\label{sec:frame_rule}
\paragraph{Frame Rule}
The Hoare rules for the heap operations are defined as local rules for the sake of simplicity. To make claims about more complex programs with effects outside just a local scope, we need to widen our perspective. We need to prove that the behaviour of a given command is not changed by the fact that there might be an additional part of the heap which the it does not access. For this type of local reasoning, we adopt the frame rule as described by Yang and O'Hearn\,\cite{Yang02asemantic}.

\begin{figure}
\[
	\infrule{
		\triple
			{P}
			{\;c\;}
			{Q}
		}
		{
		\triple
			{P\;*\;R}
			{\;c\;}
			{Q\;*\;R}
		}
\]
\begin{center}
\textit{where no variable occurring free in R is modified by c.}
\end{center}
\caption{The standard definition of the frame rule.}
\label{fig:frame_rule}
\end{figure}

As stated in the side condition in Figure \ref{fig:frame_rule}, the frame rule only holds in the event that $c$ does not modify $R$. This means that whenever this rule is used to prove a property of a program, it would have to be proven that the command does not modify $R$. To avoid this, we alter the frame rule slightly: If $c$ does not modify $R$, then $R$ is unchanged by the execution of $c$. In the case that $c$ does modify $R$, there must exist a list of values that when substituted with the variables that have been modified by $c$, will restore the original state of $R$ before $c$ was executed. We can use this to construct a postcondition for the frame rule that will preserve the side condition from the standard frame rule. We formalize the modified frame rule in Figure \ref{fig:modified_frame_rule}.

\begin{figure}
\[
	\infrule{
		\triple
			{P}
			{\;c\;}
			{Q}
		}
		{
		\triple
			{P\;*\;R}
			{\;c\;}
			{Q\;*\;(\exists vs.\;R\subst{vs}{modified\_by\;c})}
		}
\]
\caption{The modified definition of the frame rule.}
\label{fig:modified_frame_rule}
\end{figure}

\paragraph{Safety Monotonicity}
In Section \ref{sec:hoare_triple} we introduced the notion of safety to our Hoare triple, as a way of ensuring that no commands that evaluate to an erroneous state can satisfy the triple. Thus far we have only reasoned about safety in a local scope, but when using the local Hoare rules to prove programs involving arbitrary heaps, we assume safety monotonicity. This means that if executing a command c is safe in a state involving a heap $h'$ , and $h'\;\heapsubop\;h$, then c must also be safe in a state involving $h$. This property follows from the frame rule.
\todo{Is this section accurate?}
%!TEX root = ./report.tex
\paragraph{Rules of Consequence}
The hoare rules might at times might differ from the ones needed when proving a program. While they might be logically equivalent, their differences prevents us from unifying them with what we are trying to prove\todo{Ref about rules of consequence}. Hoare logic uses the rules of consequence to get around this by proving that an assertion that is stronger than another assertion can substitute it as postcondition, and an assertion that is weaker than another assertion can substitute it as precondition:

\[
	\infrule{P\entails P' \;\; \triple{P'}{\;c\;}{Q}} {\triple{P}{\;c\;}{Q}} 
	\;\;\;\;\;\; 
	\infrule{Q'\entails Q \;\; \triple{P}{\;c\;}{Q'}} {\triple{P}{\;c\;}{Q}}
\]

While these rules have not been changed by our changes to states in Imp, the change to assertions have changed how these rules are proven.
