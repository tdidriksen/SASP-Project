%!TEX root = ./report.tex
\section{Solution}
\todo{mentioned only nats here, or maybe even earlier}
\subsection{Programming with Mutable State}
Originally the state in Imp is managed by a stack, which maps variables into values. We extend this state by adding a heap, which will provide us with a mapping between addresses and value. These values represent the mutable structures. As mentioned earlier our heap only handle natural numbers.

\paragraph{}\textbf{Definition: }A heap is a mapping between an address, represented by a natural number (nat), and specific values on the heap (nat).

\begin{center}
\begin{coqdoccode}
\coqdocnoindent
\coqdockw{Definition} \coqdocvar{Heap} := \coqdocvar{Map} [ \coqdocvar{nat}, \coqdocvar{nat} ].\coqdoceol
\end{coqdoccode}

\end{center}

Section \ref{sec:heap_operations}

\subsection{Assertions with Heaps}
%!TEX root = ./report.tex
\subsection{Operations on the Heap}
\label{sec:heap_operations}
We now define syntax and operational semantics for the operations added to the Imp language for manipulating the heap. Four operations are needed: Read (or lookup), Write (or mutate), Allocate, and Deallocate (or dispose).
\paragraph{Syntax.}
The Imp syntax for each of the operations is shown in Figure \ref{fig:heap_semantics} as the first part of the \textless \textgreater -notation.
\paragraph{Functions manipulating the heap.}
\todo{Consider moving this section into 'Programming with mutable state'}
The following heap-manipulating functions underlying the implementation of the heap operations in Imp are used in the semantics presented in Figure \ref{fig:heap_semantics}. The definitions of {\it a} and {\it v} are as presented in Section \todo{Add reference to 'Programming with mutable state' here}.
\begin{itemize}
\item {\it write} $\eqdef$ {\it a} $\rightharpoonup$ {\it v} $\rightharpoonup$ {\it heap}
\item {\it dealloc} $\eqdef$ {\it a} $\rightharpoonup$ {\it heap}
\item {\it alloc} $\eqdef$ {\it a} $\rightharpoonup$ {\it n} $\rightharpoonup$ {\it heap}
\end{itemize}
The second parameter of the {\it alloc} function, {\it n}, is a natural number specifying the number of cells to allocate.

\subsubsection{Semantics}
\begin{figure}
\[
    \infrule[Read]{
       heap(a) = n
    }{
       < (X\;<\sim\; [ a ]),\;Some(s,\;h) > \; \Downarrow Some\;([s\;|\;X:n],\;h)
    }
\]
\\
\[
    \infrule[Write]{
       \exists e. heap(a) = e
    }{
       < ([ a ]\;<\sim\; b),\;Some(s,\;h) > \; \Downarrow Some\;([s\;|\;X:a],\;write\;a\;b\;h)
    }
\]
\\
\[
    \infrule[Deallocate]{
       \exists e. heap(a) = e
    }{
       < (DEALLOC\;a),\;Some(s,\;h) > \; \Downarrow Some\;(s,\;dealloc\;a\;h)
    }
\]
\\
\[
    \infrule[Allocate]{
       \neg \exists e. heap(a) = e \;\;\;\;\;\;\;\;\; \forall n.\; (n > a \wedge n \leq a+n) \implies \neg \exists e'. heap(n) = e'
    }{
       < (X=ALLOC\;n),\;Some(s,\;h) > \; \Downarrow Some\;([s\;|\;X:a],\;alloc\;a\;n\;h)
    }
\]
\caption{Semantics for the heap operations}
\label{fig:heap_semantics}
\end{figure}
The semantics for the heap operations are given in Figure \ref{fig:heap_semantics}. There are a few interesting points to note here. Firstly, the semantics of allocation and deallocation are asymmetric. This asymmetry becomes evident when the input to the two underlying functions are examined: {\it alloc} accepts a number {\it n} of cells to allocate, while {\it dealloc} accepts a single address to deallocate. Therefore, it is possible to encapsulate the allocation of an arbitrary number of memory cells in a single allocation operation, while deallocating {\it n} cells from the heap has to be done in {\it n} individual deallocation operations. This limitation makes deallocation easier to reason about, but places a heavier burden of memory management on the user of the language.

Secondly, locating free addresses in the address space for the allocation operation is not a part of the allocation semantics, although the semantics do state that these addresses must be consecutive not already allocated. \todo{Perhaps this point belongs in a different section} Here, we simply assume that such consecutive and free addresses exist in the heap, and leave it up to the soundness proof of the corresponding Hoare rule (see Section \todo{Add reference to Hoare rules section here}) to prove this fact.

Lastly, it should be noted that we have no notion of void types and do not provide any measures for ignoring evaluated values. Consequently, providing a variable in which to store the result of a read or allocation operation is required. For allocation, this also has the benefit of safeguarding against allocating unreachable memory, which would be unfortunate in an environment without garbage collection.

\todo{The address space of the heap is infinite! Write this somewhere}
\todo{We allocate cells with 0! Write this somewhere}

\subsubsection{Error Semantics}
\label{sec:error_semantics}
An interesting consequence of adding an addressable state to a programming language is the possibility of writing programs that evaluate to a faulty state. Contrary to the basic version of the Imp language, an addressable state enables the user to write programs that type check but evaluate to an erroneous state: When reading, writing, or deallocating, the program might end up in a faulty state if attempting to access an inactive address on the heap. Note that because we assume that the address space of the heap is infinite, allocation cannot fail. We present the error semantics in Figure \ref{fig:heap_error_semantics}.
\begin{figure}
\[
    \infrule[ReadError]{
       \neg \exists e. heap(a) = e
    }{
       < (X\;<\sim\; [ a ]),\;Some(s,\;h) > \; \Downarrow None
    }
\]
\\
\[
    \infrule[WriteError]{
       \neg \exists e. heap(a) = e
    }{
       < ([ a ]\;<\sim\; b),\;Some(s,\;h) > \; \Downarrow None
    }
\]
\\
\[
    \infrule[DeallocateError]{
       \neg \exists e. heap(a) = e
    }{
       < (DEALLOC\;a),\;Some(s,\;h) > \; \Downarrow None
    }
\]
\caption{Error semantics for the Read, Write, and Deallocate operations}
\label{fig:heap_error_semantics}
\end{figure}
\subsection{Definition of Hoare Triple}
%!TEX root = ./report.tex
As previously mentioned Imp uses a hoare triple to reason about the behaviour of the commands. Due to the change to assertions to now also make claims about heaps, a change to the definition of this hoare triple is needed. Initially the hoare triples way to express claims about commands was as follows:
\\
\\
\textit{If a command c is started in a state satisfying an assertion P, then the resulting state will satisfy an assertion Q, if c terminates.}
\\
\\
While this essentially still holds there are certain parameters added that the above no longer reasons about. As discussed in the secion \ref{sec:error_semantics}, some of the added commands can fail, which will result in an faulty state. This is an issue, since the hoare triple claims that resulting state is guaranteed to satisfy the post-condition, which it cannot do if it is faulty. As such we need our hoare triple to also express that the command c does not result in an error state.
\\
\\
\textit{If a command c is started in a state st satisfying an assertion P, and c does not fail in st, then the resulting state will satisfy an assertion Q, if c terminates.}
\\
\\

To express the idea of a command not failing in a state we introduce the notion of safety. A command is safe in a state if the command does not evaluate to None. We therefore  define safety as follows:

\[safe\;c\;st\;: \;\neg \;(c \;\textbackslash{}\;st\;\Downarrow None)\]

With this we can formulate a new definition of our hoare triples. As stated above the resulting state of c satisfies Q if c is safe in st. Since the case where c is not safe in st never can evaluate to true, this is expressed with conjunction rather than implication\todo{Maybe omit}. This gives us the following definition for hoare triples:

\[
\forall st.\; P\;st \impl safe\;c\;st \land \forall st'. \;c\;\textbackslash{}\;st\;\Downarrow Some\; st' \impl Q\;st'
\]	
\subsection{Hoare Rules}
\paragraph{Read}
\paragraph{Write}
\paragraph{Allocate}
\paragraph{Deallocate}
\subsection{Frame rule and Rules of Consequence}
