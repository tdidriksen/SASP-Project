%!TEX root = ./report.tex
\section{Solution}
\todo{mentioned only nats here, or maybe even earlier}
\subsection{Programming with Mutable State}
Originally the state in Imp is managed by a stack, which maps variables into values. We extend this state by adding a heap, which will provide us with a mapping between addresses and value. These values represent the mutable structures. As mentioned earlier our heap only handle natural numbers.

\paragraph{}\textbf{Definition: }A heap is a mapping between an address, represented by a natural number (nat), and specific values on the heap (nat).

\begin{center}
\begin{coqdoccode}
\coqdocnoindent
\coqdockw{Definition} \coqdocvar{Heap} := \coqdocvar{Map} [ \coqdocvar{nat}, \coqdocvar{nat} ].\coqdoceol
\end{coqdoccode}

\end{center}

Section \ref{sec:heap_operations}

\subsection{Assertions with Heaps}
%!TEX root = ./report.tex
\subsection{Operations on the Heap}
We now define syntax and operational semantics for the operations added to the Imp language for manipulating the heap. Four operations are needed: Read (or lookup), Write (or mutate), Allocate, and Deallocate (or dispose).
\subsubsection{Syntax} \todo{Show the syntax for each of the commands here. Coqdoc might be of use.}
\subsubsection{Semantics}
\todo{A figure here with the semantics for read, write, allocate, and deallocate in sexy logic notation.}
There are a few interesting points to note here. Firstly, the semantics of allocation and deallocation are asymmetric. This asymmetry becomes evident when the input to the two operations are examined: allocation accepts a (natural) number of cells to allocate, while deallocation accepts a single address. Therefore, it is possible to encapsulate the allocation of an arbitrary number of memory cells in a single allocation operation, while deallocating {\it n} cells from the heap has to be done in {\it n} individual deallocation operations. This limitation makes deallocation easier to reason about, but places a heavier burden of memory management on the user of the language.

Secondly, locating free addresses in the address space for the allocation operation is not a part of the allocation semantics, although the semantics do state that these addresses must be consecutive not already allocated. \todo{Perhaps this point belongs in a different section} Here, we simply assume that such consecutive and free addresses exist in the heap, and leave it up to the soundness proof of the corresponding Hoare rule (see Section \todo{Add reference to Hoare rules section here}) to prove this fact.

Lastly, it should be noted that we have no notion of void types and do not provide any measures for ignoring evaluated values. Consequently, providing a variable in which to store the result of a read or allocation operation is required. For allocation, this also has the benefit of safeguarding against allocating unreachable memory, which would be unfortunate in an environment without garbage collection.

\todo{The address space of the heap is infinite! Write this somewhere}
\todo{We allocate cells with 0! Write this somewhere}

\subsubsection{Error Semantics}
An interesting consequence of adding a an addressable state to a programming language is the possibility of writing programs that evaluate to a faulty state. Contrary to the basic version of the Imp language, an addressable state enables the user to write programs that type check but evaluate to an erroneous state: When reading, writing, or deallocating, the program might end up in a faulty state if attempting to access an inactive address on the heap. Note that because we assume that the address space of the heap is infinite, allocation cannot fail. \todo{Add some more juice to this section.}
\subsection{Definition of Hoare Triple}
%!TEX root = ./report.tex
\subsection{Definition of the Hoare Triple}
\label{sec:hoare_triple}
As previously mentioned, specifications of Imp programs are given using Hoare triples. Due to the new assertion type presented in Section \ref{sec:assertions_heaps}, a change of the definition of this hoare triple is needed accordingly. The basic Hoare triple is defined as follows:
\\
\\
\textit{If a command c is started in a state satisfying an assertion P, then the resulting state will satisfy an assertion Q, if c terminates.}\todo{Add reference to definition.}
\\
\\
While this essentially still holds for the extended Imp specifications, this definition is insufficient. As discussed in Section \ref{sec:error_semantics}, some of the heap-manipulating operations can evaluate to a faulty state. This is an issue, since the Hoare triple claims that the resulting state is guaranteed to satisfy the post-condition. With the above definition, this can no longer be guaranteed if the resulting state is faulty. As such, we need our Hoare triple to  express that the command c does not result in an error state.
\\
\\
\textit{If a command c is started in a state st satisfying an assertion P, and c does not fail in st, then the resulting state will satisfy an assertion Q, if c terminates.}
\\
\\
To express the idea of a command not failing in a state, we introduce the notion of safety. A command is safe in a state {\it st} if the command, starting in {\it st}, does not evaluate to None. We therefore define safety as follows:

\[safe\;c\;st\;: \;\neg \;(<c,\;st>\;\Downarrow None)\]

Having formalized the a safety predicate, we can now formulate a new definition of our Hoare triple. This gives us the following definition of a Hoare triple:

\[
\forall st.\; P\;st \impl safe\;c\;st \land \forall st'. <c,\;st>\;\Downarrow Some\; st' \impl Q\;st'
\]	
\subsection{Hoare Rules}
\paragraph{Read}
\paragraph{Write}
\paragraph{Allocate}
\paragraph{Deallocate}
\subsection{Frame rule and Rules of Consequence}
