%!TEX root = ./report.tex
\section{Verifying Imp Programs}
\todo{Introduction to proving - maybe with why and methodology}
An interesting property of the Imp language is that any Imp program can be formally verified directly within the Coq environment. This section introduces Hoare logic, the formal system within which Imp programs are verified, together with its main concepts. Additionally, we define the rules required for reasoning about the operations of the extended Imp language.
\subsection{Hoare Logic}
Hoare logic\,\cite{Hoare69anaxiomatic} is a formal system for program verification. It formalizes the evaluation of a program using two types of assertions: preconditions and postcondtions. A precondition is an assertion that holds prior to program execution, and a postcondition is an assertion that holds after program execution.

In Hoare logic, only partial correctness of a program is proven. This means that we only prove the correctness of a program in the event that the program terminates. Proving that a program terminates is a separate activity, which we will not consider here.
%partial correctness
\subsubsection{The Hoare Triple}
The Hoare triple is one of the central concepts of Hoare logic.

\defbf{Original Hoare Triple\label{original_hoare}} \textit{If a command c is started in a program state satisfying an assertion P, and if c eventually terminates, then the resulting program state is guaranteed to satisfy the assertion Q}\,\cite{Pierce:SF}\textit{.}

\paragraph{}
Note that in the case that $c$ does not terminate, we can not say anything about Q. Below is a formalization of Definition 7.

\[
\forall st\;st'.\;\langle c,\;st\rangle\;\Downarrow Some\; st' \impl P\;st \impl Q\;st'
\]

The addition of a heap to the program state, however, renders the above formalization insufficient. As discussed in Section \ref{sec:error_semantics}, some of the heap-manipulating operations can evaluate to a faulty program state. This is an issue, since the Hoare triple claims that if $c$ terminates, the resulting state is guaranteed to satisfy the post-condition. With the above definition, this can no longer be guaranteed if the resulting program state is faulty. As such, we need our Hoare triple to express that the command $c$ does not evaluate to an erroneous program state.

\defbf{Redefined Hoare Triple} \textit{If a command c is started in a program state st satisfying an assertion P, if c eventually terminates, and if c does not fail in st, then the resulting program state is guaranteed to satisfy the assertion Q. We use the following notation for the Hoare triple: } $\triple{P}{\;c\;}{Q}$

\paragraph{}
To express the idea of a command not failing in a given program state, we introduce the notion of safety. A command is safe in a state {\it st} if the command, starting in {\it st}, does not evaluate to $None$. We therefore define safety as follows:

\[ \mathit{safe\;c\;st\;: \;\neg \;(\langle c,\;st\rangle\;\Downarrow None)}\]

Having formalized the safety predicate, we can now formalize Definition 8 as follows:

\[
\forall st.\; P\;st \impl safe\;c\;st \land \forall st'. \langle c,\;st\rangle\;\Downarrow Some\; st' \impl Q\;st'
\]

Any reference to $the\;Hoare\;triple$ in subsequent sections will refer to the Hoare triple as defined in Definition 8.

\subsubsection{Hoare Rules --- Part 1}
The Hoare rules for the operations found in the basic version of Imp are shown in Figure \ref{fig:hoare_rules_basic_imp}. Because these are all defined in ``Software Foundations'', we will not go through the justification for each of them. As an example, however, we will consider the justification for the Assignment rule.

Let us think of the $Q$ in the Assignment rule as any property of natural numbers, such as ``being equal to 1'', and note the fact that $a$ is an arithmetic expression on natural numbers. This means that if we have $Q(a)$, that $Q$ holds for $a$, we can state that ``$a$ is equal to 1''. Then the assignment command, $X ::= a$, transfers this property of $a$ to $X$, so that we are now able to state that ``$X$ is equal to 1'', or $Q(X)$. Rephrasing this intuition, we can say that if we substitute all occurrences of $X$ in $Q$ with $a$ in the precondition, and we know that $Q(a)$ holds, then when we transfer the property of $Q$ to $X$, then $Q$ must hold in the postcondition, since $Q(a)$ held in the precondition.

Taking a closer look at the assignment rule, we realize that it only holds for assigning a variable to the state, and not the heap. In fact, we have no way of expressing concise propositions about the heap inside a Hoare rule. For stating propositions about the heap, we turn to separation logic.
\begin{figure}
\[
    \infrule[Assignment]{}{
      	\triple
      		{Q\subst a X} 
      		{\;X::=a\;}
	  		{Q}
    }
    \;\;\;\;\;\;\;\;
    \infrule[Skip]{}{
      	\triple
      		{P} 
      		{\; \mathbf{SKIP} \;}
	  		{P}
    }
\]

\[
    \infrule[Sequence]{
    	\triple
      		{P} 
      		{\;c1\;}
	  		{Q}
	  		\;\;
	  	\triple
      		{Q} 
      		{\;c2\;}
	  		{R}
    }{
      	\triple
      		{P} 
      		{\;c1;c2\;}
	  		{R}
    }
    \;\;\;\;\;\;\;\;
    \infrule[While]{
    	\triple
      		{P \land b} 
      		{\; c \;}
	  		{P}
    }{
      	\triple
      		{P} 
      		{\; \mathbf{WHILE}\;{b}\;\mathbf{DO}\;{c}\;\mathbf{END} \;}
	  		{P \land \neg b}
    }
\]
\[
    \infrule[If]{
    	\triple
      		{P \land b} 
      		{\; c1 \;}
	  		{Q}
	  	\;\;
	  	\triple
      		{P \land \neg b} 
      		{\; c2 \;}
	  		{Q}
    }{
      	\triple
      		{P} 
      		{\;\mathbf{IFB}\;{b}\;\mathbf{THEN}\;{c1}\;\mathbf{ELSE}\;{c2}\;\mathbf{FI} \;}
	  		{Q}
    }
\]
\caption{Hoare rules of the Imp language.}
\label{fig:hoare_rules_basic_imp}
\end{figure}
\todo{Revise title of section}
\todo{Mention partial correctness here.}

\subsection{Separation Logic}
\label{sec:separation_logic}
Separation logic is an extension of Hoare logic that enables local reasoning about shared resources. While its original purpose was to provide assertions for reasoning about shared mutable data structures, such as heaps, it has now been applied to a broader range of subject areas, such as shared-variable concurrency\,\cite{reynolds2008AnIntroductionTo}\todo{Add page number?}.

\subsubsection{The Connectives}
For separation logic, three new connectives are added to those already known from Hoare logic. 
\paragraph{The separating conjunction.}
The most characteristic connective of separation logic is the {\it separating conjunction}, denoted by {\it $\ast$}. A separating conjunction {\it p $\ast$ q} on a heap {\it h} is a spatial connective, which states that {\it h} can be split into two separate (or disjoint) parts, one for which {\it p} holds and one for which {\it q} holds. Note that {\it p $\ast$ q} holds for the entirety of {\it h}, meaning that there is no third part of {\it h} for which a third predicate {\it r} would hold.

\paragraph{{\it emp}.}
{\it emp} is the unit of the separating conjunction, and only holds for the empty heap. Consequently, since {\it emp} only holds for the empty heap, {\it p $\ast$ emp} is logically equivalent to {\it p}.

\paragraph{The magic wand.}
Despite its name, there is nothing magic about the magic wand. Also known as the {\it separating implication}, {\it p $\wand$ q} says that if a heap is extended with a separate part for which p holds, then q holds for that extended heap.

\paragraph{The points-to relation.}
Aside from the new connectives, separation logic uses the {\it points-to} relation. Denoted by {\it e} $\mapsto$ {\it e'}, it states that {\it e} points to {\it e'} on the heap, and nothing else is on the heap. As such, the points-to relation both says something about the contents and the size of the heap. Given {\it x} $\mapsto$ {\it x'} $\ast$ {\it y} $\mapsto$ {\it y'}, we know that in one part of the heap, {\it x} points to {\it x'}, and in another part of the heap, {\it y} points to {\it y'}, and nothing else is on the heap.

\subsubsection{Local Reasoning and the Frame Rule}
One of the key points of separation logic is the ability to reason locally about a shared mutable resource, which in our case is a heap. Local reasoning is achieved by specifying a program in terms of the heap cells  it accesses, and then using the frame rule to abstract away the rest of the heap. For this, we adopt the frame rule as described by Yang and O'Hearn\,\cite{Yang02asemantic}.

\begin{figure}
\[
	\infrule{
		\triple
			{P}
			{\;c\;}
			{Q}
		}
		{
		\triple
			{P\;*\;R}
			{\;c\;}
			{Q\;*\;R}
		}
\]
\begin{center}
\textit{where no variable occurring free in R is modified by c.}
\end{center}
\caption{The standard definition of the frame rule.}
\label{fig:frame_rule}
\end{figure}

If we have a program {\it C} with the precondition {\it P} and the postcondition {\it Q}, we can abstract away the rest of the heap {\it R}, as we know that {\it c} does not modify any part of {\it R}.


\subsubsection{Separation Algebra}
\defbf{Separation Algebra}
\textit{A separation algebra is a partial, commutative monoid ($\Sigma$, $\circ$, {\it u}) where $\Sigma$ is the carrier type, $\circ$ is the binary operator, and {\it u} is the unit element}\,\cite{Calcagno07:LCS}.

\paragraph{}
We can define heaps in terms of a separation algebra by taking $\Sigma$ to be a type of heaps, $\circ$ to be a composition operator for two disjoint heaps, and {\it u} to be {\it emp}, the unit of the separating conjunction \cite{BirkedalL:veroop-conf}.

\todo{Have another look at forall and exists axioms}
\begin{figure}
\begin{center}
\setlength{\tabcolsep}{0.5cm}
\begin{tabular}{ c c c }
$
	\infrule{}
		{
		  A \vdash \top
		}
$ & $
	\infrule{}
		{
		  \bot \vdash A
		}
$ & $
	\infrule
	    {
			A\;x \vdash B
	    }
		{
			\forall A, A \vdash B
		}
$ \\ \\
$
	\infrule[]
	    {
			\forall x, B \vdash A\;x
	    }
		{
			\forall A, B \vdash A
		}
$ &
$
	\infrule[]
	    {
			\forall x, A\;x \vdash B
	    }
		{
			\exists A, A \vdash B
		}
$ &
$
	\infrule[]
	    {
			B \vdash A\;x
	    }
		{
			\exists A, B \vdash A
		}
$ \\ \\
$
	\infrule[]
	    {
			A \vdash C
	    }
		{
			A \land B \vdash C
		}
$ &
$
	\infrule[]
	    {
			B \vdash C
	    }
		{
			A \land B \vdash C
		}
$ &
$
	\infrule[]
	    {
			A \vdash B
	    }
		{
			A \vdash B \lor C
		}
$ \\ \\
$
	\infrule[]
	    {
			A \vdash C
	    }
		{
			A \vdash B \lor C
		}
$ &
$
	\infrule[]
	    {
			A \vdash C\;\;\;B \vdash C
	    }
		{
			A \lor B \vdash C
		}
$ &
$
	\infrule[]
	    {
			A \vdash B\;\;\;A \vdash C
	    }
		{
			A \vdash B \land C
		}
$ \\ \\
\multicolumn{3}{c}{
$
	\infrule[]
	    {
			A \vdash (B \Rightarrow C)
	    }
		{
			A \land B \vdash C
		}
	\;\;\;\;\;\;\;\;\;\;\;\;\;\;\;\;
	\infrule[]
	    {
			A \land B \vdash C
	    }
		{
			A \vdash (B \Rightarrow C)
		}
$}

\end{tabular}
\end{center}

% ltrueR: forall C, C |-- ltrue;
% lfalseL: forall C, lfalse |-- C;
% lforallL: forall T x (P: T -> Frm) C, P x |-- C -> lforall P |-- C;
% lforallR: forall T (P: T -> Frm) C, (forall x, C |-- P x) -> C |-- lforall P;
% lexistsL: forall T (P: T -> Frm) C, (forall x, P x |-- C) -> lexists P |-- C;
% lexistsR: forall T x (P: T -> Frm) C, C |-- P x -> C |-- lexists P;
% landL1: forall P Q C, P |-- C  ->  P //\\ Q |-- C;
% landL2: forall P Q C, Q |-- C  ->  P //\\ Q |-- C;
% lorR1:  forall P Q C, C |-- P  ->  C |-- P \\// Q;
% lorR2:  forall P Q C, C |-- Q  ->  C |-- P \\// Q;
% landR:  forall P Q C, C |-- P  ->  C |-- Q  ->  C |-- P //\\ Q;
% lorL:   forall P Q C, P |-- C  ->  Q |-- C  ->  P \\// Q |-- C;
% landAdj: forall P Q C, C |-- (P -->> Q) -> C //\\ P |-- Q;
% limplAdj: forall P Q C, C //\\ P |-- Q -> C |-- (P -->> Q)

\caption{The axioms of our intuitionistic logic.}
\label{fig:axioms_int_logic}
\end{figure}

\begin{figure}
\begin{center}
\setlength{\tabcolsep}{0.5cm}
\begin{tabular}{ c c c }
$
	\infrule{}
		{
 		  P * Q \vdash Q * P
		}
$ & $
	\infrule{}
		{
 		  (P * Q) * R \vdash P * (Q * R)
		}
$ & $
	\infrule
	    {
			P * Q \vdash R
	    }
		{
			P \vdash Q \wand R
		}
$ \\ \\
$
	\infrule[]
	    {
			P \vdash Q \wand R
	    }
		{
			P * Q \vdash R
		}
$ &
$
	\infrule[]
	    {
			P \vdash Q
	    }
		{
			P * R \vdash Q * R
		}
$ &
$
	\infrule[]
		{}
	    {
			P * emp \vdash P
	    }
$ \\ \\
&
$
	\infrule[]
		{}
	    {
			P \vdash P * emp
	    }
$ &
\end{tabular}
\end{center}

\caption{The axioms of our separation logic.}
\label{fig:axioms_sep_logic}
\end{figure}

\subsubsection{Building the Separation Logic}
\todo{Maybe add a reference to low-level separation logic-paper}
This section describes how we have used prior work by Jesper Bengtson et al.\,\cite{BirkedalL:veroop-conf} to build a separation logic. Here, an intuitionistic logic is defined as a logic for which the axioms presented in Figure \ref{fig:axioms_int_logic} hold. Accordingly, a separation logic is defined as a logic for which the axioms presented in Figure \ref{fig:axioms_sep_logic} hold.

\paragraph{Purpose.}
The purpose of building the separation logic is to be able to create assertions {\it heap} $\to$ {\it state} $\to$ {\it Prop} for use in program specifications that need to make claims both a state and a heap.

\paragraph{Procedure.} 
The basis of our separation logic is {\it Prop}, the sort for propositions in Coq\,\cite{CoqIntro}. It is established that Prop is an intuitionistic logic by proving that it satisfies all the axioms of an intuitionistic logic. To further develop the intuitionistic logic of {\it Prop}, we exploit the following properties of intuitionistic logic:

\begin{enumerate}[label=\arabic*)]
\item We can to create an intuitionstic logic from a function space {\it A} $\to$ {\it B}, where {\it B} is the carrier of an intuitionistic logic and {\it A} is any type.
\item We can create a separation logic from a function space {\it A} $\to$ {\it B}, where {\it A} is the carrier of a separation algebra and {\it B} is the carrier of an intuitionistic logic.\,\cite{JBSlides}
\end{enumerate} \todo{maybe move 2) to between the next two paragraphs.}

By 1), we can extend the intuitionistic logic of {\it Prop} with a state to achieve an intermediate logic {\it state} $\to$ {\it Prop}. This will allow us to create assertions about programs using a state. Note that all of the connectives defined for the logic of {\it Prop} now work on assertions of {\it state} $\to$ {\it Prop}, as these are just pointwise liftings.

As we have already established that we can define heaps in terms of a separation algebra, we extend {\it state} $\to$ {\it Prop} with a type of heaps by 2). Thus we achieve a separation logic \heap $\to$ {\it state} $\to$ {\it Prop}, with which we can create assertions about programs involving both a state {\it and} a heap. Furthermore, since we create the separation logic from an equivalence relation on heaps, we can create assertions that are closed under the equivalence of heaps. This means that any assertion that holds for a given heap {\it h} will also hold for any equivalent heap {\it h'}. Analogously to the intermediate logic, the connectives now work on assertions of \heap $\to$ {\it state} $\to$ {\it Prop}.

\paragraph{}
Using this logic, we can now define Hoare rules in terms of assertions that make claims about heaps. In other words, we are now able to define Hoare rules for the heap-manipulating operations of the extended Imp language. 

\todo{We define our heap to be a partial function on natural numbers mapping addresses (locations) to values}

\subsubsection{Hoare Rules --- Part 2}
\todo{Revise title of section}
\label{sec:hoare_rules_heap}
% All the rules have been proven sound in Coq.
This section describes Hoare rules for each of the four heap-manipulating operations. Note that these rules are local\,\cite{Reynolds02}, and thus rely on the frame rule for reasoning about non-local specifications.
\subsubsection{Read}
\[
	\infrule[Read]{}
		{
		\triple
			{e\mapsto e'}
			{\;X\;\lsquigarr\;\lbrack\;e\;\rbrack\;}
			{\exists v.\;(e\;\mapsto\;e')\subst{v}{X}\;\land\;(X=(e'\subst{v}{X}))}
		}
\]
The Read rule reads the value at address {\it e}, {\it e'}, into the state. Because the value of {\it X} has now changed, all previous occurrences of {\it X} must be substituted for the old value of {\it X}, here denoted by the existential variable {\it v}, to avoid corrupting previous definitions.

\subsubsection{Write}
\[
	\infrule[Write]{}
		{
		\triple
			{e\mapsto-}
			{\;\lbrack\;e\;\rbrack\;\lsquigarr\;e'\;}
			{e\mapsto e'}
		}
\]
The Write rule destructively updates an address on the heap. Importantly, the precondition requires the updated address to be active, i.e. it must exist on the heap beforehand. If one wishes to write to a non-active address, the address must be allocated first.
\todo{Write somewhere that all addresses are evaluated on the stack.}

\subsubsection{Allocate}
\[
	\infrule[Allocate]{}
		{
		\triple
			{\;emp\;}
			{\;X\;\&\!\!\!=ALLOC\;n\;}
			{\exists a. X=a\;\land\;\odot _{i=a}^{a+n-1} i\mapsto0}
		}
\]
The Allocate rule allocates {\it n} memory cells as specified by the parameter to {\it ALLOC}, and lets the variable {\it X} point to the first of these cells. As it is the case in other programming languages, such as Java\,\cite{JavaDataTypes}, we allocate memory cells with a default value of zero. We do not have a separate notion of a {\it null} pointer, so pointing to zero is equivalent to pointing to {\it null}. Interpreting a value of zero as either a concrete `0' or a {\it null} pointer is up to the program.

The existential variable $a$ in the postcondition denotes the address at which allocation begins. Because of our assumption that the address space of the heap is infinite, we can always choose this $a$ to be greater than any other active address on the heap. Hence, allocation never fails.
% We allocate with 0
% We can always find an address

\subsubsection{Deallocate}
\[
	\infrule[Deallocate]{}
		{
		\triple
			{e\mapsto-}
			{\;DEALLOC\;e\;}
			{emp}
		}
\]
The Deallocate rule removes an active address from the heap. As reflected in the semantics presented in Section \ref{sec:semantics}, deallocation is asymmetric with respect to allocation.

\subsection{The Frame Rule}
\label{sec:frame_rule}
To make claims about more complex programs with effects outside just a local scope, we need to widen our perspective. We need to prove that the behaviour of a given command is not changed by the fact that there might be an additional part of the heap which it does not modify. 

The side condition of the frame rule states that the frame rule only holds in the event that $c$ does not modify $R$. This means that in order to use this rule, it would have to be proven that the command does not modify $R$. To avoid this, we alter the frame rule slightly: If $c$ does not modify $R$, then $R$ is unchanged by the execution of $c$. In the case that $c$ does modify $R$, there must exist a list of values that when substituted with the variables that have been modified by $c$, will restore the original program state of $R$ before $c$ was executed. We can use this to construct a postcondition for the frame rule that will preserve the side condition from the standard frame rule. We formalize the modified frame rule in Figure \ref{fig:modified_frame_rule}.

\begin{figure}
\[
	\infrule{
		\triple
			{P}
			{\;c\;}
			{Q}
		}
		{
		\triple
			{P\;*\;R}
			{\;c\;}
			{Q\;*\;(\exists vs.\;R\subst{vs}{modified\_by\;c})}
		}
\]
\caption{The modified definition of the frame rule.}
\label{fig:modified_frame_rule}
\end{figure}

\paragraph{Safety Monotonicity}
In Section \ref{sec:hoare_triple} we introduced the notion of safety to our Hoare triple, as a way of ensuring that no commands that evaluate to an erroneous program state can satisfy the triple. Thus far we have only reasoned about safety in a local scope, so to use the frame rule to globalize our local Hoare rules we have to consider safety monotonicity. Safety monotonicity means that if executing a command $c$ is safe in a program state involving a heap $h'$ , and $h'\;\heapsubop\;h$, then $c$ must also be safe in a program state involving $h$. To be able to prove the frame rule, we have to also prove safety monotonicity.

\subsection{Rules of Consequence}
The precondition or postcondition of a Hoare rule might at times differ from the ones needed when verifying a program. While the condition we are trying to prove might be logically equivalent to one in a given rule, their differences prevent them from being unified. To accommodate this difference, Hoare logic introduces the rules of consequence\,\cite{Hoare69anaxiomatic}. They state that any precondition $P$ can be substituted for a weaker precondition $P'$, while any postcondition $Q$ can be substituted for a stronger postcondition $Q'$. These rules are shown in Figure \ref{fig:rules_of_consequence}.

\begin{figure}
\[
	\infrule{P\vdash P' \;\; \triple{P'}{\;c\;}{Q}} {\triple{P}{\;c\;}{Q}} 
	\;\;\;\;\;\; 
	\infrule{Q'\vdash Q \;\; \triple{P}{\;c\;}{Q'}} {\triple{P}{\;c\;}{Q}}
\]
\caption{The rules of consequence.}
\label{fig:rules_of_consequence}
\end{figure}
