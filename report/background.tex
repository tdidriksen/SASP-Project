%!TEX root = ./report.tex
\section{Background}
\subsection{Coq}
Coq is a interactive theorem prover for managing formal proofs. It is a functional programming language that implements the program specification language Gallina. Due to Gallinas base in CiC (Calculus of Inductive Constructions), this provides both higher-order logics and a richly-typed functional programming language to Coq.

\subsection{Imp}
As previously mentioned, Imp is a simple imperative programming language studied in the electronic book ``Software Foundations'' \footnote{Not to be confused with Irons IMP language or the Edinburgh IMP language.} implemented in Coq. The language includes some core features of widely used programming languages such as C, and certain properties about the language as a whole have been proven. This helps program verification, as the precise definition of Imp can be used to formally prove that a certain program satisfies particular specifications of their behavior. 

The language has variables which can be of two types, natural numbers and booleans. These can be assigned both from values or other variables, and both come with a set of operators. Operators available for natural numbers are addition (+), subtraction (-), multiplication (*), and division (/), and the boolean operators are equality (==), inequality (<=), negation (not), and conjunction (and). Furthermore the control flow statements if and while are also available and proven.

The behavior of each of the aforementioned commands is specified with hoare triples. %More about hoare?
As with the rest of the project, Imp deals with partial correctness. It is important to note that the Imp languages worked on in this project is the one defined up to the Hoare Logic chapter\footnote{http://www.itu.dk/courses/SASP/F2013/Hoare.html}, thus things like the type checker is not included.
\subsection{Separation Algebra}
Separation Logic.
 - Separation Algebra
 - Partial Correctness