%!TEX root = ./report.tex
\section{The Imp Language}
\label{sec:background_imp}
% As previously mentioned, Imp\footnote{Not to be confused with Irons IMP language or the Edinburgh IMP language.} is a simple imperative programming language studied in the electronic book ``Software Foundations'' and implemented in Coq. The language includes a selection of core features of widely used imperative programming languages such as C and Java, \todo{"... and certain properties about the language as a whole have been proven" lyder underligt} and certain properties about the language as a whole have been proven. This helps verification of programs written in Imp, as the precise definition of Imp can be used to formally prove that a certain program satisfies a particular specification. 
In this section we will describe the Imp language and its features. The parts of Imp originating from its initial implementation will be presented\,\cite{Pierce:SF} along side our additions. It is worth noting that expressions in Imp are limited to arithmetics and booleans.\todo{Seems odd here, maybe move to grammar section?}

\subsection{The Program State}
To represent values at a point of execution Imp uses a $state$. This state, is a partially applied function $id \rightharpoonup value$\todo{Give a definition of id?}, where $id$ identifies the variable. The language has a command for assignment of variables, along with control-flow statements for conditionals, loops, and sequencing of commands. To be able to reason about these commands, assertions are used to make claims about a particular state of a program during execution. Given a state these assertions will give a claim based on this state\todo{Maybe example of this?}. As such assertions are defined as $state \rightharpoonup proposition$. Each of the command is specified by a Hoare rule, and will be discussed further in section \ref{sec:hoare_rules}.
\todo{Revise this section. Should be about the old state and the addition of the heap.}

%To be able to implement such programs as list reversal in Imp\todo{Something wrong.}, we need to introduce the idea of a mutable state. Originally, the state in Imp is managed by a stack, which is a function from variables to values.
We extend this state by adding a heap, which will provide us with a mapping from addresses to values. For the sake of simplicity, we will consider the address space of the heap to be infinite, which means that a program cannot run out of memory. We define heaps as a partial function from addresses ($nat$) to values ($nat$):

\[
\heap \eqdef a\rightharpoonup v, \; where\; a, \;v \in \mathbb{N}
\] 

\subsection{Grammar}

\begin{figure}
\begin{alltt}
                  program = com ;
                  com =   "SKIP"
                        | id "::=" aexp            (* ASSIGNMENT *)
                        | com ";" com
                        | "IFB" bexp "THEN" com "ELSE" com "FI"
                        | "WHILE" bexp "DO" com END
                        | id \lsquigarr "[" aexp "]"       (* READ *)
                        | "[" aexp "]" \lsquigarr id       (* WRITE *)
                        | "DEALLOC" aexp           (* DEALLOCATE *)
                        | id "&=" "ALLOC" nat ;    (* ALLOCATE *)
                  aexp =  "ANum" nat
                        | "AId" id
                        | "APlus" aexp aexp
                        | "AMinus" aexp aexp
                        | "AMult" aexp aexp ;
                  bexp =  "BTrue"
                        | "BFalse"
                        | "BEq" aexp aexp
                        | "BLe" aexp aexp
                        | "BNot" bexp
                        | "BAnd" bexp bexp ;
                  id = "Id" nat ;
                  nat = natural number, 0 included ;
\end{alltt}
\label{fig:imp_grammar}
\caption{Grammar for the extended Imp language.}
\end{figure}

The full grammar for the extended Imp language, including operations for manipulating the heap, namely Read, Write, Deallocate, and Allocate, are shown in Figure \ref{imp_grammar}. Besides these, we have assignment, sequencing, conditional statements, and while loops. 

\subsection{Semantics}
\label{sec:semantics}
We now define syntax and operational semantics for the operations of the Imp language. Four operations have been added for manipulating the heap: Read (or lookup), Write (or mutate), Allocate, and Deallocate (or dispose).

The Imp syntax for each of the operations is shown in Figure \ref{fig:heap_semantics} as the first part of the $\langle \; \rangle$  -notation. \todo{What is this notation?}
\paragraph{Functions manipulating the heap.}
\todo{Consider moving this section into 'Programming with mutable state'}
The following heap-manipulating functions underlying the implementation of the heap operations in Imp are used in the semantics presented in Figure \ref{fig:heap_semantics}. The definitions of {\it a} and {\it v} are as presented in Section \ref{sec:programming_with_mutable_state}.
\begin{itemize}
\item {\it write} $\eqdef$ {\it a} $\rightharpoonup$ {\it v} $\rightharpoonup$ {\it heap}
\item {\it dealloc} $\eqdef$ {\it a} $\rightharpoonup$ {\it heap}
\item {\it alloc} $\eqdef$ {\it a} $\rightharpoonup$ {\it n} $\rightharpoonup$ {\it heap}
\end{itemize}
The second parameter of the {\it alloc} function, {\it n}, is a natural number specifying the number of cells to allocate.

\subsubsection{Semantics}
\todo{Explain the notation for the semantics}
\begin{figure}

\[
    \infrule[Skip]{}
    {
       \langle \mathbf{SKIP},\;Some(s,\;h) \rangle \; \Downarrow Some(s,\;h)
    }
\]
\\
\[
    \infrule[Assignment]{}
    {
       \langle (\cassign{X}{n}),\;Some(s,\;h) \rangle \; \Downarrow Some(update\;s\;X\;n,\;h)
    }
\]
\\
\[
    \infrule[Sequence]{
      \langle c1,\;st \rangle \; \Downarrow Some(st')\;\;\;\;
      \langle c2,\;st' \rangle \; \Downarrow Some(st'')
    }
    {
       \langle c1;c2,\;st \rangle \; \Downarrow Some(st'')
    }
\]
\\
\[
    \infrule[If True]{
      b = true\;\;\;\;
      \langle c1,\;st \rangle \; \Downarrow Some(st')
    }
    {
       \langle (\;\mathbf{IFB}\;{b}\;\mathbf{THEN}\;{c1}\;\mathbf{ELSE}\;{c2}\;\mathbf{FI}),\;st \; \rangle \; \Downarrow Some(st')
    }
\]
\\
\[
    \infrule[If False]{
      b = false\;\;\;\;
      \langle c2,\;st \rangle \; \Downarrow Some(st')
    }
    {
       \langle (\;\mathbf{IFB}\;{b}\;\mathbf{THEN}\;{c1}\;\mathbf{ELSE}\;{c2}\;\mathbf{FI}),\;st \; \rangle \; \Downarrow Some(st')
    }
\]
\\
\[
    \infrule[While Loop]{
      b = true\;\;\;\;
      \langle c,\;st \rangle \; \Downarrow Some(st')
    }
    {
       \langle (\;\mathbf{WHILE}\;{b}\;\mathbf{DO}\;{c}\;\mathbf{END}),\;st \; \rangle \; \Downarrow Some(st')
    }
\]
\\
\[
    \infrule[While End]{
      b = false
    }
    {
       \langle (\;\mathbf{WHILE}\;{b}\;\mathbf{DO}\;{c}\;\mathbf{END}),\;st \; \rangle \; \Downarrow Some(st)
    }
\]
\\
\[
    \infrule[Read]{
       heap(a) = n
    }{
       \langle  (X\;<\sim\; [ a ]),\;Some(s,\;h) \rangle \; \Downarrow Some\;([s\;|\;X:n],\;h)
    }
\]
\\
\[
    \infrule[Write]{
       \exists e. heap(a) = e
    }{
       \langle  ([ a ]\;<\sim\; b),\;Some(s,\;h) \rangle \; \Downarrow Some\;([s\;|\;X:a],\;write\;a\;b\;h)
    }
\]
\\
\[
    \infrule[Deallocate]{
       \exists e. heap(a) = e
    }{
       \langle  (DEALLOC\;a),\;Some(s,\;h) \rangle \; \Downarrow Some\;(s,\;dealloc\;a\;h)
    }
\]
\\
\[
    \infrule[Allocate]{
       \neg \exists e. heap(a) = e \;\;\;\;\;\;\;\;\; \forall n.\; (n > a \wedge n \leq a+n) \implies \neg \exists e'. heap(n) = e'
    }{
       \langle (X=ALLOC\;n),\;Some(s,\;h) \rangle \; \Downarrow Some\;([s\;|\;X:a],\;alloc\;a\;n\;h)
    }
\]
\caption{Semantics for the operations in Imp}
\label{fig:imp_semantics}
\end{figure}
The semantics for the heap operations are given in Figure \ref{fig:heap_semantics}. There are a few interesting points to note here. Firstly, the semantics of allocation and deallocation are asymmetric. This asymmetry becomes evident when the input to the two underlying functions are examined: {\it alloc} accepts a number {\it n} of cells to allocate, while {\it dealloc} accepts a single address to deallocate. Therefore, it is possible to encapsulate the allocation of an arbitrary number of memory cells in a single allocation operation, while deallocating {\it n} cells from the heap has to be done in {\it n} individual deallocation operations. This limitation makes deallocation easier to reason about, but places a heavier burden of memory management on the user of the language.

Secondly, locating free addresses in the address space for the allocation operation is not a part of the allocation semantics, although the semantics do state that these addresses must be consecutive and   not already allocated. \todo{Perhaps this point belongs in a different section} Here, we simply assume that such consecutive and free addresses exist in the heap, and leave it up to the soundness proof of the corresponding Hoare rule (see Section \ref{sec:hoare_rules}) to prove this fact.

Lastly, it should be noted that we have no notion of void types and do not provide any measures for ignoring evaluated values. Consequently, providing a variable in which to store the result of a read or allocation operation is required. For allocation, this also has the benefit of safeguarding against allocating unreachable memory, which would be unfortunate in an environment without garbage collection.

\todo{The address space of the heap is infinite! Write this somewhere}
\todo{We allocate cells with 0! Write this somewhere}

\subsubsection{Error Semantics}
\label{sec:error_semantics}
An interesting consequence of adding an addressable state to a programming language is the possibility of writing programs that evaluate to a faulty state. Contrary to the basic version of the Imp language, an addressable state enables the user to write programs that type check but evaluate to an erroneous state: When reading, writing, or deallocating, the program might end up in a faulty state if attempting to access an inactive \todo{Inactive? Not invalid, or something else?} address on the heap. Note that because we assume that the address space of the heap is infinite, allocation cannot fail. We present the error semantics in Figure \ref{fig:heap_error_semantics}.
\begin{figure}
\[
    \infrule[ReadError]{
       \neg \exists e. heap(a) = e
    }{
       < (X\;<\sim\; [ a ]),\;Some(s,\;h) > \; \Downarrow None
    }
\]
\\
\[
    \infrule[WriteError]{
       \neg \exists e. heap(a) = e
    }{
       < ([ a ]\;<\sim\; b),\;Some(s,\;h) > \; \Downarrow None
    }
\]
\\
\[
    \infrule[DeallocateError]{
       \neg \exists e. heap(a) = e
    }{
       < (DEALLOC\;a),\;Some(s,\;h) > \; \Downarrow None
    }
\]
\caption{Error semantics for the Read, Write, and Deallocate operations}
\label{fig:heap_error_semantics}
\end{figure}