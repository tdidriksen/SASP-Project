%!TEX root = ./report.tex
\section{The Imp Language}
\label{sec:background_imp}
% As previously mentioned, Imp\footnote{Not to be confused with Irons IMP language or the Edinburgh IMP language.} is a simple imperative programming language studied in the electronic book ``Software Foundations'' and implemented in Coq. The language includes a selection of core features of widely used imperative programming languages such as C and Java, \todo{"... and certain properties about the language as a whole have been proven" lyder underligt} and certain properties about the language as a whole have been proven. This helps verification of programs written in Imp, as the precise definition of Imp can be used to formally prove that a certain program satisfies a particular specification. 
In this section we describe the Imp language and its features. The parts of Imp originating from its initial implementation will be presented\,\cite{Pierce:SF} alongside our additions. We will refer to the unmodified version of Imp as \textit{the basic version of Imp}, and the version including our modifications as \textit{the extended version of Imp}.

\subsection{The Program State}
\label{sec:the_program_state}
Values at a point of execution Imp are represented by a $state$. In both the basic and the extended version of Imp, this state is a partially applied function $id \to value$, where $id$ identifies a variable. 

\defbf{State} \textit{The state is a function from identifiers to values, which is defined as }$state \eqdef id \to value.$

\paragraph{}
We extend the storage in Imp with a mutable state in the form of a heap. For modeling this heap, we use maps from the Coq Containers library\,\cite{CoqContainers}, which allow us to map addresses to values. For the sake of simplicity, we will consider the address space of the heap to be infinite, which means that a program cannot run out of memory. 

\defbf{Heap} \textit{The heap is a partial function mapping addresses to values, which is specified } $\heap \eqdef a\to v, \; where\; a, \;v \in \mathbb{N}.$

\paragraph{} 
To avoid terminological confusion, Definition 1 will henceforth be referred to as $the\;state$, Definition 2 as $the\;heap$, and the combination of the two as $the\;program\;state$.

\paragraph{Functions operating on the heap.}
For lookup on the heap we will be using the existing $find$ function from the implementation of maps\,\cite{CoqContainers}. Find returns a value wrapped in Coq's standard Option type\,\cite{CoqOption}, meaning that if the key exists in the map it will return $Some(value)$, otherwise it will return $None$.

\defbf{Find} \textit{Find is a function which given a key k and a map m, returns the value at key k wrapped in an option type. Find is defined as }{\it find} $\eqdef$ {\it key} $\to$ {\it map} $\to$ {\it option value.}

\paragraph{}
In our case, the key and value are natural numbers, because of our definition of the heap.
\paragraph{}
The following heap-manipulating functions underlying the implementation of the heap operations in Imp are used for specifying the semantics discussed in Section \ref{sec:semantics}.

\defbf{write}\textit{write is a function which given a value v, an address a, and a heap h, returns h, extended with a mapping from a to v, such that lookup on a results in v. Write is defined as} {\it write} $\eqdef$ {\it a} $\to$ {\it v} $\to$ {\it heap} $\to$ {\it heap.}
\defbf{dealloc} \textit{dealloc is a function which given an address a and a heap h, returns h, with no mapping from a to v, such that lookup on a fails. Deallocate is defined as }{\it dealloc} $\eqdef$ {\it a} $\to$ {\it heap} $\to$ {\it heap.}
\defbf{alloc} \textit{alloc is a function which given an address a, a natural number of cells to allocate n, and a heap h, returns h extended with a mapping from a to 0, a+1 to 0} $\cdots$ \textit{a+n-1 to 0, such that lookup on a, a+1} $\cdots$ \textit{a+n-1 all result in 0. allocate is defined as }{\it alloc} $\eqdef$ {\it a} $\to$ {\it n} $\to$ {\it heap} $\to$ {\it heap.}

\subsection{Grammar}

\begin{figure}
\begin{alltt}
                  program = com ;
                  com =   "SKIP"
                        | id "::=" aexp            (* ASSIGNMENT *)
                        | com ";" com
                        | "IFB" bexp "THEN" com "ELSE" com "FI"
                        | "WHILE" bexp "DO" com END
                        | id \lsquigarr "[" aexp "]"       (* READ *)
                        | "[" aexp "]" \lsquigarr id       (* WRITE *)
                        | "DEALLOC" aexp           (* DEALLOCATE *)
                        | id "&=" "ALLOC" nat ;    (* ALLOCATE *)
                  aexp =  "ANum" nat
                        | "AId" id
                        | "APlus" aexp aexp
                        | "AMinus" aexp aexp
                        | "AMult" aexp aexp ;
                  bexp =  "BTrue"
                        | "BFalse"
                        | "BEq" aexp aexp
                        | "BLe" aexp aexp
                        | "BNot" bexp
                        | "BAnd" bexp bexp ;
                  id = "Id" nat ;
                  nat = natural number, 0 included ;
\end{alltt}
\caption{Full grammar for the extended Imp language, specified in EBNF.}
\label{fig:imp_grammar}
\end{figure}

The full grammar for the extended Imp language, including the added operations for manipulating the heap, namely read, write, deallocate, and allocate, is shown in Figure \ref{fig:imp_grammar}. Besides these, we have skip, assignment, sequencing, conditional statements, and while loops from the basic version of Imp. Basically, any Imp program consists of a command, arithmetic expressions on natural numbers, and boolean expressions. All variables, implemented as identifiers on natural numbers ({\it id}), are evaluated with the {\it state}. Since the domain of the heap is the natural numbers, an address on the heap is accessed by evaluating any {\it aexp} representing a memory address to a natural number with the state.

\subsection{Semantics}

\label{sec:semantics}
We now define operational semantics for the operations of the Imp language. Four operations have been added for manipulating the heap: Read, Write, Allocate, and Deallocate.

\paragraph{Notation.}
The operational semantics are specified using the following notation:

\[
    \infrule[Title]{
      Preconditions
    }
    {
       \langle (c),\;(s,\;h) \rangle \Downarrow Some\;(s',\;h')
    }
\]

Read this notation as follows: Given that the preconditions hold (if any), then evaluating the command $c$ in a program state with a state $s$ and heap $h$ will result in a program state with a state $s'$ and a heap $h'$. Due to the error semantics explained later in this section, the resulting program state is wrapped in an Option type\,\cite{CoqOption}. Thus, any faulty program state is represented by $None$, while any normal program state is represented by $Some\;(state,\;heap)$.

Updating a state is described using this notation:
\begin{center}$[s\;|\;X:n]$\end{center}
This means that the state $s$ is updated with a variable $X$ with the value $n$. If $X$ already exists in the state, the new value for $X$ will shadow the existing value.

Lastly, evaluating a boolean is described using this notation:

\begin{center}$b\Downarrow BTrue \;\;\;\;\;\;\; b\Downarrow BFalse$\end{center}
The left one means that $b$ evaluates to $true$, and the right one that $b$ evaluates to $false$.

\subsubsection{Semantics}
\begin{figure}

\[
    \infrule[Skip]{}
    {
       \langle \mathbf{SKIP},\;(s,\;h) \rangle \Downarrow Some\;(s,\;h)
    }
\]
\\
\[
    \infrule[Assignment]{}
    {
       \langle (X\;::=\;n),\;(s,\;h) \rangle \Downarrow Some\;([s\;|\;X:n],\;h)
    }
\]
\\
\[
    \infrule[Sequence]{
      \langle c1,\;(s,\;h) \rangle \Downarrow Some\;(s',\;h')\;\;\;\;
      \langle c2,\;(s',\;h') \rangle \Downarrow Some\;(s'',\;h'')
    }
    {
       \langle c1;c2,\;(s,\;h) \rangle \Downarrow Some\;(s'',\;h'')
    }
\]
\\
\[
    \infrule[IfTrue]{
      b\Downarrow BTrue\;\;\;\;\;\;\;\;
      \langle c1,\;(s,\;h) \rangle \Downarrow Some\;(s',\;h')
    }
    {
       \langle (\mathbf{IFB}\;{b}\;\mathbf{THEN}\;{c1}\;\mathbf{ELSE}\;{c2}\;\mathbf{FI}),\;(s,\;h) \rangle \Downarrow Some\;(s',\;h')
    }
\]
\\
\[
    \infrule[IfFalse]{
      b\Downarrow BFalse\;\;\;\;\;\;\;\;
      \langle c2,\;(s,\;h) \rangle \Downarrow Some\;(s',\;h')
    }
    {
       \langle (\mathbf{IFB}\;{b}\;\mathbf{THEN}\;{c1}\;\mathbf{ELSE}\;{c2}\;\mathbf{FI}),\;(s,\;h) \rangle \Downarrow Some\;(s',\;h')
    }
\]
\\
\[
    \infrule[WhileLoop]{
      \begin{array}{c}
      b\Downarrow BTrue\;\;\;\;\;\;
      \langle c,\;(s,\;h) \rangle \Downarrow Some\;(s',\;h') \\
      \langle (\mathbf{WHILE}\;{b}\;\mathbf{DO}\;{c}\;\mathbf{END}),\;(s',\;h') \rangle \Downarrow Some\;(s'',\;h'')
      \end{array}
    }
    {
       \langle (\mathbf{WHILE}\;{b}\;\mathbf{DO}\;{c}\;\mathbf{END}),\;(s,\;h) \rangle \Downarrow Some\;(s'',\;h'')
    }
\]
\\
\[
    \infrule[WhileEnd]{
      b\Downarrow BFalse
    }
    {
       \langle (\mathbf{WHILE}\;{b}\;\mathbf{DO}\;{c}\;\mathbf{END}),\;(s,\;h) \rangle \Downarrow Some\;(s,\;h)
    }
\]
\\
\[
    \infrule[Read]{
       heap(a) = n
    }{
       \langle  (X\;\lsquigarr\; [ a ]),\;(s,\;h) \rangle \Downarrow Some\;([s\;|\;X:n],\;h)
    }
\]
\\
\[
    \infrule[Write]{
       \exists e. heap(a) = e
    }{
       \langle  ([ a ]\;\lsquigarr\; b),\;(s,\;h) \rangle \Downarrow Some\;([s\;|\;X:a],\;write\;a\;b\;h)
    }
\]
\\
\[
    \infrule[Deallocate]{
       \exists e. heap(a) = e
    }{
       \langle  (\mathbf{DEALLOC}\;a),\;(s,\;h) \rangle \Downarrow Some\;(s,\;dealloc\;a\;h)
    }
\]
\\
\[
    \infrule[Allocate]{
       \neg \exists e. heap(a) = e \;\;\;\;\;\;\;\;\; \forall n'.\; (n' > a \wedge n' \leq a+n) \impl \neg \exists e'. heap(n') = e'
    }{
       \langle (X\;\&\!\!\!=\mathbf{ALLOC}\;n),\;(s,\;h) \rangle \Downarrow Some\;([s\;|\;X:a],\;alloc\;a\;n\;h)
    }
\]
\caption{Semantics for the operations in Imp.}
\label{fig:imp_semantics}
\end{figure}
For completeness, the semantics for all the operations of the extended Imp language are given in Figure \ref{fig:imp_semantics}. We will, however, not discuss the semantics of the operations already existing in the basic version of Imp, as these have not been altered. A discussion of these can be found in ``Software Foundations''\,\cite{Pierce:SF}.

Regarding the operations on the heap, there are a few interesting points to note. Firstly, the semantics of allocation and deallocation are asymmetric. This asymmetry becomes evident when we examine the semantics of the two operations more closely. The parameter $a$ of the Deallocate operation represents an arithmetic expression denoting the address of the cell we wish to deallocate. In contrast, the parameter $n$ of the Allocate operation denotes the number of cells we wish to allocate on the heap. Hence, we can allocate $n$ cells with one invocation of Allocate, but deallocating $n$ cells requires $n$ invocations of Deallocate. This asymmetry makes deallocation easier to reason about, but places a heavier burden of memory management on the user of the language.

Secondly, locating free addresses in the address space for the allocation operation is not a part of the allocation semantics, although the semantics do state that these addresses must be consecutive and not already allocated. Here, we simply assume that such consecutive and free addresses exist on the heap. The problem of finding such addresses is addressed by the soundness proof of the corresponding Hoare rule for allocation, presented in Section \ref{sec:hoare_rules_heap}.

Lastly, it should be noted that we have no notion of void types and do not provide any measures for ignoring evaluated values. Consequently, providing a variable in which to store the result of a Read or Allocate operation is required. For Allocate, this also has the benefit of safeguarding against allocating unreachable memory, which would be unfortunate in an environment without garbage collection.

\subsubsection{Error Semantics}
\label{sec:error_semantics}
An interesting consequence of adding an addressable heap to a programming language is the possibility of writing programs that evaluate to a faulty program state. Contrary to the basic version of the Imp language, an addressable heap enables the user to write programs that type check, but evaluate to an erroneous program state: When reading, writing, or deallocating, the program might end up in a faulty program state if attempting to access an inactive address on the heap. Note that because we assume that the address space of the heap is infinite, allocation cannot fail. The error semantics are presented in Figure \ref{fig:heap_error_semantics}.
\begin{figure}
\[
    \infrule[ReadError]{
       \neg \exists e. heap(a) = e
    }{
       \langle (X\;\lsquigarr\;[ a ]),\;Some\;(s,\;h) \rangle \Downarrow None
    }
\]
\\
\[
    \infrule[WriteError]{
       \neg \exists e. heap(a) = e
    }{
       \langle ([ a ]\;\lsquigarr\; b),\;Some\;(s,\;h) \rangle \Downarrow None
    }
\]
\\
\[
    \infrule[DeallocateError]{
       \neg \exists e. heap(a) = e
    }{
       \langle (\mathbf{DEALLOC}\;a),\;Some\;(s,\;h) \rangle \Downarrow None
    }
\]
\caption{Error semantics for the Read, Write, and Deallocate operations.}
\label{fig:heap_error_semantics}
\end{figure}