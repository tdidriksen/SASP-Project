%!TEX root = ./report.tex
\subsection{Hoare Rules for the Heap Operations}
\label{sec:hoare_rules}
% All the rules have been proven sound in Coq.
This section describes Hoare rules for each of the four operations presented in Section \ref{sec:heap_operations}. All of the rules have been proven sound in Coq. Note that these rules are local\,\cite{Reynolds02}, and thus rely on the frame rule (see Section \ref{sec:frame_rule}) for reasoning about non-local specifications.
\subsubsection{Read}
\[
	\infrule[Read]{}
		{
		\triple
			{e\mapsto e'}
			{\;X<\sim\lbrack\;e\;\rbrack\;}
			{\exists v.\;(e\;\mapsto\;e')\subst{v}{X}\;\land\;(X=e')\subst{v}{X}}
		}
\]
The Read rule reads the value at address {\it e}, {\it e'}, onto the stack. Consequently, all previous occurrences of {\it X} must be substituted with the old value of {\it X}, here denoted by the existential variable {\it v}, to avoid corrupting previous definitions.

\subsubsection{Write}
\[
	\infrule[Write]{}
		{
		\triple
			{e\mapsto-}
			{\;\lbrack\;e\;\rbrack\;<\sim\;e'\;}
			{e\mapsto e'}
		}
\]
The Write rule destructively updates an address on the heap. Importantly, the precondition requires the updated address to be active, i.e. it must exist on the heap beforehand. If one wishes to write to a non-active address, the address must be allocated first.
\todo{Write somewhere that all addresses are evaluated on the stack.}

\subsubsection{Allocate}
\[
	\infrule[Allocate]{}
		{
		\triple
			{\;emp\;}
			{\;X\;\&=ALLOC\;n\;}
			{\exists a. X=a\;\land\;\odot _{i=a}^{a+n-1} i\mapsto0}
		}
\]
The Allocate rule allocates {\it n} memory cells as specified by the parameter to {\it ALLOC}, and lets the variable {\it X} point to the first of these cells. As it is the case in other programming languages, such as Java\,\cite{JavaDataTypes}, we allocate memory cells with a default value of zero. We do not have a separate notion of a {\it null} pointer, so pointing to zero is equivalent to pointing to {\it null}. Interpreting a value of zero as either a concrete `0' or a {\it null} pointer is up to the program.
% We allocate with 0
% We can always find an address

\subsubsection{Deallocate}
\[
	\infrule[Deallocate]{}
		{
		\triple
			{e\mapsto-}
			{\;DEALLOC\;e\;}
			{emp}
		}
\]
The Deallocate rule removes an active address from the heap. As reflected in the semantics presented in Section \ref{sec:heap_operations}, deallocation is asymmetric with respect to allocation.
