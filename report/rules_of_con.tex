%!TEX root = ./report.tex
\paragraph{Rules of Consequence}
The precondition or postcondition of a Hoare rule might at times differ from the ones needed when proving a program. While the condition in our goal might be logically equivalent to one in a given rule, their differences prevents them from being unified. To accommodate this difference, Hoare logic introduces the rules of consequence\,\cite{Hoare69anaxiomatic}. They state that any precondition $P$ can be substituted for a weaker precondition $P'$, provided that $P$ entails $P'$, while any postcondition $Q$ can be substituted for a stronger postcondition $Q'$, provided that $Q'$ entails $Q$. These rules are shown in Figure \ref{fig:rules_of_consequence}.

\begin{figure}
\[
	\infrule{P\entails P' \;\; \triple{P'}{\;c\;}{Q}} {\triple{P}{\;c\;}{Q}} 
	\;\;\;\;\;\; 
	\infrule{Q'\entails Q \;\; \triple{P}{\;c\;}{Q'}} {\triple{P}{\;c\;}{Q}}
\]
\caption{The rules of consequence.}
\label{fig:rules_of_consequence}
\end{figure}

While these rules have not been changed by our changes to states in Imp, the change to assertions have changed how these rules are proven.